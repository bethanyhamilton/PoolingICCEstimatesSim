
We only estimated variance components of the primary studies in each simulated meta-analysis using REML. However, researchers use a variety of estimation methods when estimating variance components and will not always report which method they implemented. As stated before, estimating variance components accurately is essential for getting accurate ICC estimates (see Equation \ref{ICC_algebra}), so pooled estimates will only be as accurate as the ICC estimates that are being pooled. Future studies might investigate the impact of various estimation methods for the variance components themselves on which each ICC estimate is based. 
    
We also investigated scenarios in which each primary study only reported a single ICC estimate. The reality is that studies often report results for multiple outcomes, and therefore, multiple ICC values might be available per primary study. These outcomes may be related and so will be the resulting ICC estimates. The researcher might want to include all the ICC estimates reported for the related outcomes when pooling ICC estimates. Luckily, RVE is typically used to account for dependence in meta-analytic data sets, so this method should be robust to this limitation. The work accomplished in this study could be extended to evaluate methods for pooling of ICC estimates in scenarios with within-study dependence in the ICCs.  

There are several limitations to the variance formulae derived for an ICC estimate for reasonable use in meta-analysis of ICC estimates. The formulae derived by \citeA{donner1980a} and \citeA{smith1957} both require knowledge of how many individuals there are per cluster in a primary study. However, typically primary studies do not report how many individuals there are per cluster.  Also, the $v_r$ formula derived by \citeA{hedgesVarianceIntraclassCorrelations2012} requires knowledge of the variance of the cluster-level variance in a primary study. This is also not often reported in primary studies which drastically limits a researcher ability to use this formula unless they have access to the primary study's raw data. Furthermore, some of the formulae require the weighted mean cluster size \cite{smith1957, swiger1964} instead of the arithmetic mean cluster size. However, it is not likely that primary studies will report the weighted mean cluster size nor be clear about how the average cluster size was calculated if it is reported. Further research is needed to evaluate the use of the mean versus a weighted mean cluster size in the relevant variance formulae. 

Furthermore, we only evaluated pooling of ICC estimates for ICCs for two-level data. While in educational, and social science research more generally, three-level data are also commonly analyzed. Future research, could evaluate use of these methods with three-level data. Only \cite{hedgesVarianceIntraclassCorrelations2012} has derived a formula for the variance of an ICC for clustered data with three levels of nesting. 

In summary, we recommend not summarizing across ICCs that come from samples that are too different. The additional obstacle of including ICCs from studies that are capturing variability in outcomes that deviate too much from the primary construct of interest is a further threat to the validity of the resulting inferences due to generalizability error. In addition, researchers are cautioned against use of ICCs when they are based on a sample from a population that is too different than the population that is the focus of the relevant analysis (meta-analysis of effect sizes or power analyses for clustered data) \cite{jacob2010}. However, under the conditions we evaluated, it seems reasonable to use Fisher's $v_r$ with RVE or REML to pool ICC estimates, rather than relying on the more ad hoc methods that are typically used for imputing values.
