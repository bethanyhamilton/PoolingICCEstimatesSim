
The cluster randomized trial (CRT) is an experimental design that involves randomly assigning entire clusters of individuals (such as classrooms of students or hospitals of patients) into the treatment conditions (\citeauthorNP{raudenbushHierarchicalLinearModels2002},  \citeyearNP{raudenbushHierarchicalLinearModels2002}; \citeauthorNP{donnerClusterRandomizationTrials2000}, \citeyearNP{donnerClusterRandomizationTrials2000}; \citeauthorNP{klarCurrentFutureChallenges2001}, \citeyearNP{klarCurrentFutureChallenges2001}; \citeauthorNP{murrayDesignAnalysisGrouprandomized1998}, \citeyearNP{murrayDesignAnalysisGrouprandomized1998}). The intraclass (or intracluster) correlation coefficient (ICC; \citeauthorNP{hedgesEffectSizesClusterRandomized2007}, \citeyearNP{hedgesEffectSizesClusterRandomized2007}; \citeauthorNP{hedgesEffectSizesThreeLevel2011}, \citeyearNP{hedgesEffectSizesThreeLevel2011}) captures the degree of clustering in data. ICC estimates are necessary for several statistical techniques that are used to handle clustered data including CRTs. In meta-analysis, the estimated ICC is necessary to adjust effect sizes from studies with clustered data. However, primary studies with clustered data do not always report ICC estimates, requiring meta-analysts to impute ICC values from other studies with similar design characteristics. Similarly, in an a priori power analysis, reasonable ICC estimates are needed if the planned study's experimental design involves clustered data like a CRT. 

ICC estimates are necessary when clustered data are encountered in both meta-analysis and a priori power analysis because the standard error of a treatment effect in clustered data is inflated by the design effect which is itself a function of the ICC, $\rho$. Even if the size of the ICC estimate (and thus the degree of cluster-dependence) is small, if the average size of a cluster is large then the ICC estimate has a major effect on the standard error of the treatment effect \cite{cornfield1978}. The accuracy of both types of clustered data analyses depends on the accuracy of the ICC estimates used. Because raw data are not available for use in prospective power analyses nor in meta-analyses, researchers typically have to impute reasonable values from ICC estimates reported in related, prior studies. 

To address the need for accurate ICC values, large secondary databases of design parameters have been used to offer reasonable ICC values for a set of research designs, outcomes, and population groups \cite{hedgesIntraclassCorrelationValues2007a, hedges2013, hedbergReferenceValuesWithinDistrict2014, adams2004}. Unfortunately, such databases, and thus ICC values, are not comprehensive for all designs, outcomes, and populations. In addition, even if the ICCs captured in those datasets were based on very similar studies, the sampling distribution of ICC estimates is unknown which makes it unclear how to best pool together ICC estimates from such databases. 

Several methods have been proposed to provide ICC values for a priori power analysis or meta-analysis with clustered primary study data. Some applied researchers have taken the arithmetic mean or median of a set of ICC estimates from a similar population context \cite{puzioDifferentiatedLiteracyInstruction2020a, grahamEffectsWritingLearning2020}. Others have constructed confidence intervals for the ICC parameter and proposed that applied researchers use the upper limit as a conservative estimate of the ICC \cite{donner1986}. In the case of power analysis, this method would produce a sample size larger than needed and could result in wasted resources \cite{donner1986}. Finally, some have applied Bayesian methods to create a posterior distribution for the ICC estimates \cite{spiegelhalter2001, turnerAllowingImprecisionIntracluster2004, turnerPriorDistributionsIntracluster2005}. These methods have relied on strong assumptions about ICC estimates' distribution and its moments. Unfortunately, and as noted, the sampling distribution for ICC estimates is unknown and any empirical distribution of ICC estimates drawn from samples have typically been highly right-skewed \cite{hedbergReferenceValuesWithinDistrict2014, stockford2009}. In addition, simple averages of ICCs do not capture the differing precisions of the ICC estimates. ICC estimates based on larger sample sizes will be more precise than others based on smaller sample sizes. When calculating an average ICC estimate, more weight should be assigned to more precise ICC estimates. 

Thus, another alternative for imputing a reasonable ICC estimate can involve use of principled meta-analytic methods that are more typically used for pooling effect sizes. When pooling effect sizes, a meta-analyst can calculate a weighted average effect size where weights are a direct function of precision. Most typically, the inverse of the effect size's variance is used as the effect size's weight such that the more precise the estimate, the smaller its standard error and the more weight it is afforded \cite{tanner-smith2014}. Similarly, a researcher could pool ICC estimates from different studies' datasets and assign more weight in the pooling to more precise ICC estimates. However, accurate use of meta-analytic pooling requires knowledge of the shape and variance of the sampling distribution of the effect size of interest (here, of ICC estimates). While many different formulae for the variance of an ICC estimate have been proposed \cite{smith1957, swiger1964,hedgesVarianceIntraclassCorrelations2012,fisherTheoryStatisticalEstimation1925,donner1980, fisher1970statistical}, it is unclear which formulation of the variance of an ICC might offer the best weight formulation when pooling ICC estimates to obtain an estimate of the population ICC. The aim of this study is to compare methods for quantitatively synthesizing ICC values to obtain the best pooled ICC estimate for use in a priori power analyses and when adjusting effect size estimates in meta-analysis for primary studies involving clustered data.

There are several choices for how to pool effect sizes (for this study, ICCs) using meta-analysis. Two options for random effects pooling of effect sizes include method of moments estimation in the robust variance estimation framework (RVE; \cite{hedges2010}) and restricted maximum likelihood estimation (REML; \cite{viechtbauer2015}). RVE is robust to the choice of distributional form of the effect size being pooled and should be less impacted by the effect size weights used in the meta-analytic model \cite{hedges2010}. Because the distributional form and sampling variance of ICC estimates are unknown, a meta-analytic technique like RVE seems well matched for meta-analysis of ICC estimates. For comparison, we also evaluated the use of REML estimation with random effect pooling because it is still a commonly used estimation procedure in meta-analysis. In addition to comparing estimation procedures, We also investigated how well different formulas for the variance of the ICC distribution performed in the inverse variance weights used when pooling ICC estimates. This study is intended to provide guidelines for obtaining unbiased pooled ICC estimates that ultimately can be used in multiple scenarios including a priori power analyses and meta-analysis of effect size estimates for scenarios with clustered data.





