There is a need for using accurate ICC values for a priori power analyses or meta-analyses of effect sizes in clustered data scenarios. The most frequently used methods for imputing ICC estimates in these scenarios can result in inaccurate values that ultimately negatively impact the inferences resulting from the associated analyses in which the values are used. Little research has focused on how best to capture reasonable ICC values to improve the validity of associated inferences. Meta-analytic pooling of relevant observed ICC values offers a principled method for imputing ICC values. However, there has not been a lot of research exploring how best to meta-analyze ICC estimates. The unknown shape of the sampling distribution of ICC estimates offers a particular challenge to the meta-analytic pooling of ICC estimates. In addition, the formulation for the variance of the sampling distribution of ICC estimates is unknown although several methodological researchers have suggested various formulas \cite{fisherTheoryStatisticalEstimation1925, hedgesVarianceIntraclassCorrelations2012, donner1980a, smith1957, swiger1964, fisher1970statistical}. The current study compares some of the variance formulas (used in the weighting of ICCs when they are pooled) and compares two methods for meta-analytic pooling (REML versus RVE) across a number of design conditions. Ultimately, this study is intended to assess the meta-analytic pooling of ICC to inform researchers about the best way to do so.  

The results of this study provide some guidelines for applied researchers when (1) they need to impute ICC values for use in an a priori power analysis for clustered data, or (2) the researcher is conducting a meta-analysis that contains effect sizes for clustered data in any of the primary studies in the meta-analytic dataset that need to be corrected for clustering and that might be missing associated ICC estimates, or (3) the researcher is meta-analyzing ICC estimates. The results suggest that, for the design conditions we examined, there were no huge differences in the pooled ICC estimates when using REML versus using RVE to pool ICC estimates. Therefore, when meta-analyzing ICCs in the kinds of scenarios that we examined, REML and RVE appear to function equivalently. There were not huge differences among variance formulae, but the Fisher TF formula (Equation \ref{fisher_transformed_ICC_var}; \citeA{fisher1970statistical}) performed best across the performance criteria we assessed and the design conditions we tested. The Fisher TF formula provided the least biased pooled ICC estimates and SE of the ICC estimates for the $\rho=0.05$ conditions, and it was the only $v_r$ formula without substantial bias in the pooled ICC estimates in the $\rho = 0.15$ conditions.  So, we recommend using the Fisher TF formula for the ICC variance for use in the inverse variance weights when pooling ICC estimates. For large $\rho$ values ($\rho=0.25$), no bias was found as a function of the $v_r$ formula used. 

Regarding primary study characteristics, larger sample sizes resulted in less biased pooled ICC estimates across all $v_r$ formulae. Last, we did not find any distinctive impact on ICC estimation of the number of ICC estimates being pooled (k= 20, 50, versus 100). Future studies could evaluate how well pooling fewer ICC estimates captures the true ICC parameter value.  

Finally, these meta-analytic pooling ICC estimates methods are relatively accessible to applied researchers. For example, RVE for meta-analysis is available in software like Stata, SPSS, and R. These pooling methods are already being used widely by applied meta-analysts. We provide sample code that also facilitates the calculation of the ICC variance using several of the options in our supplementary materials. We also provide the code for meta-analysis of ICC estimates.
