
In summary, thus far, the methods that have been proposed to estimate an ICC from external sources (primary studies) are less than ideal. Primarily, the methods each include one or more of the four limitations: (1) a reliance on availability of raw data, (2) requirement of strong assumptions of distributional form for ICC estimates, (3) potentially biased/imprecise ICC estimates, and (4) overly conservative ICC estimates. One potential method for addressing these short-comings is a pooling technique typically applied to meta-analysis of effect sizes.  

Limited work has been done to evaluate meta-analytic estimation methods used when pooling ICCs. In particular, best practice for specifying weights requires additional attention and guidance. Meta-analytic weights are typically a function of the effect size's sampling error variance. Because multiple different formulations for the ICC's variance have been suggested, research on the optimal formulation for this variance is also needed.   

The current study aims to address the following research question: How well does pooling ICC estimates capture the population ICC value 
a) under a variety of conditions when assuming a random effects model paired with
b) RVE versus REML estimation and when 
c) both estimation procedures are used with different ICC estimate variance formulae in the inverse variance weights?


